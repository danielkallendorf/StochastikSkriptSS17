\documentclass[10pt,a4paper]{article}

\usepackage{luatex85}
\def\pgfsysdriver{pgfsys-pdftex.def}

%\usepackage[utf8]{inputenc}
%\usepackage{fontenc}
\usepackage[utf8]{luainputenc}

\usepackage[german]{babel}
\usepackage{amsmath}
\usepackage{amsfonts}
\usepackage{amssymb}
\usepackage{amsthm}
\usepackage{graphicx}
\usepackage{tikz,pgf}
\usetikzlibrary{cd}
\usetikzlibrary{babel}
\usepackage{mathrsfs}
\usepackage{framed}
\usepackage{ulem}
\usepackage{tabularx}
\usepackage{csquotes}
\usepackage{dsfont}
\usepackage{enumitem}
\usepackage[hidelinks]{hyperref}



\newcommand{\N}{\ensuremath{\mathbb{N}}}
\newcommand{\Z}{\ensuremath{\mathbb{Z}}}
\newcommand{\Q}{\ensuremath{\mathbb{Q}}}
\newcommand{\R}{\ensuremath{\mathbb{R}}}
\newcommand{\C}{\ensuremath{\mathbb{C}}}
\newcommand{\F}{\ensuremath{\mathbb{F}}}
\newcommand{\AF}{\ensuremath{\mathbb{A}}}

\newcommand{\la}{\ensuremath{\lambda}}
\newcommand{\al}{\ensuremath{\alpha}}
\newcommand{\ol}[1]{\overline{#1}}
\newcommand{\ul}[1]{\underline{#1}}
\newcommand{\todomark}[1]{\fbox{\Large Hier könnte \sout{Ihre Werbung} #1 stehen}}
\newcommand{\norm}[1]{\left|#1\right|}
\newcommand{\mapsfrom}{\ensuremath{\mathrel{\reflectbox{\mapsto}}}}
\newcommand{\potset}{\mathscr P}
\newcommand{\Prb}{\mathbb P}
\newcommand{\Bor}{\mathscr B}
\newcommand{\Idop}{\mathds{1}}

\theoremstyle{definition}
\newtheorem{definition}[subsection]{Definition}
\newtheorem{prop}[subsection]{Proposition}

\theoremstyle{plain}
\newtheorem{lem}[subsection]{Lemma}
\newtheorem{kor}[subsection]{Korollar}
\newtheorem{satz}[subsection]{Satz}
\newtheorem{theorem}[subsection]{Theorem}
\newtheorem{rem}[subsection]{Erinnerung}
\newtheorem*{rem*}{Erinnerung}

\theoremstyle{remark}
\newtheorem{bem}[subsection]{Bemerkung}
\newtheorem*{bem*}{Bemerkung}
\newtheorem{exm}[subsection]{Beispiel}
\newtheorem*{exm*}{Beispiel}

\newcounter{exmlistitem}
\newenvironment{exmlist}
	{\let\oldss\thesubsection
	\renewcommand{\thesubsection}{\oldss.\arabic{exmlistitem}}
	\stepcounter{subsection}
	\let\oldexm\exm
	\let\oldendexm\endexm
	\renewenvironment{exm}{\addtocounter{subsection}{-1}\stepcounter{exmlistitem}\oldexm}{\oldendexm}}
	{\setcounter{exmlistitem}{0}}


\title{Einführung in die Stochastik}
\author{}

\begin{document}
	\maketitle
	\tableofcontents
	\newpage
%VL 18.04.2017
	
	\section{Grundlagen}
	\begin{rem}Naive Grundidee der Modellierung des Zufalls:\\
		\\
	\begin{tabularx}{\textwidth}{X|c|c}
		Konzept & mathematisches Objekt & Symbol\\ \hline
		\enquote{Alle denkbaren Ergebnisse eines zufälligen Geschehens} & Menge & $\Omega$\\
		Wahrscheinlichkeit,dass $\omega\in\Omega$ beobachtet wird & Abbildung $\Omega\to[0,1]$ &\\
		Alle denkbaren Ja-Nein-Fragen, die zum zufälligen Geschehen gestellt werden können & Potenzmenge von $\Omega$ & $\potset(\Omega)$\\
		Wahrscheinlichkeit, dass die zu $a\in\potset(\Omega)$ gehörige Frage mit ``ja'' beantwortet wird. & Abbildung $\potset(\Omega)\to [0,1]$ & $\Prb$
	\end{tabularx}
	\end{rem}
	
	\begin{exmlist}
		\begin{exm}[6-Seitiger Würfel]
			$\Omega=\{1,2,3,4,5,6\}$ und $\foreach \i in {1,...,6} {p(\i)=}\frac{1}{6}$.\\
			Ein Beispiel für eine Ja-Nein-Frage: \enquote{Ist die gewürfelte Zahl durch $3$ teilbar?} dann ist $A\in\potset(\Omega):A=\{3,6\}$ und $\Prb(A)=\frac{1}{3}$.
		\end{exm}
		\begin{exm}
			Speziell zufälllige natürliche Zahl: $\Omega=\N$, $p(1)=\frac{1}{2},p(2)=\frac{1}{4},...,p(n)=2^{-n}$.\\
			Dann gilt $\sum_{\omega=1}^{\infty}p(\omega)=1$.
			\begin{enumerate}
				\item Ja-Nein-Frage: \enquote{Ist die Zahl gerade?}\\
				Zugelassenen Menge: $A=\{2,3,6,...\}$
				\[\Prb(A)=\sum_{j=1}^{\infty}p(2j)=\sum_{j=1}^{\infty}2^{-2j}=\frac{1}{1-\frac{1}{4}}-1=\frac{1}{3}\]
				\item Ja-Nein-Frage: \enquote{Ist die Zahl Primzahl?}\\
				Zugelassene Menge: $B=\{n\in\N\mid\text{$n$ ist prim}\}$
				\[\Prb(B)=\sum_{j\in B}p(j)=???\]
				Abschätzung $\Prb(B)\leq 1-\Prb(A)+\Prb(\{2\})\leq \frac{2}{3}+\frac{1}{4}$
			\end{enumerate}
		\end{exm}
	\end{exmlist}
	
	\begin{definition}
		Sei $\Omega$ eine abzählbare Menge.\\
		Eine Abbildung $p:\Omega\to[0,1]$ mit $\sum_{\omega\in\Omega}p(\omega)=1$ heißt \textbf{Zähldichte} oder \textbf{Wahrscheinlichkeitsdichte} auf $\Omega$.
	\end{definition}

	\begin{definition}
		Man nennt dann $\Omega$ den \enquote{Ergebnisraum}, die \enquote{Grundmenge} oder \enquote{Grundgesamtheit}.\\
		Ein spezielles $A\in\potset(\Omega)$ nennt man \enquote{Ergebnis} und falls $A=\{\omega\}$ \textbf{Elementarereignis}.
	\end{definition}

	\begin{definition}
		Sei $\Omega$ eine abzählbare Menge und $\potset(\Omega)$ die Potenzmenge, $p$ sei eine Zähldichte.\\
		Dann heißt die Abbildung 
		\[\Prb:\potset(\Omega)\to[0,1],A\mapsto \sum_{\omega\in A}p(\omega)\]
		 das von $p$ erzeugte \textbf{Wahrscheinlichkeitsmaß} (kurz W-Maß).
	\end{definition}
	
	\begin{bem}
		Beachte: $p$ wird in der Notation unterdrückt. Alternativ schreibe $\Prb_p$.\\
		Außerdem: Statt $\Prb(\{\omega\})$ wird oft $\Prb(\omega)$ geschrieben.
	\end{bem}
	
	\begin{lem}
		Sei $p$ eine Zähldichte auf $\Omega$. Das von $p$ erzeugte Wahrscheinlichkeitsmaß hat folgende Eigenschaften:
		\begin{enumerate}
			\item $\Prb(\Omega)=1$
			\item Falls $(A_n)_{n\in\N}$ eine Folge von paarweise disjunkten Ereignissen ist, dann ist $\Prb(\bigcap_{n=1}^\infty A_n)=\sum_{n=1}^{\infty}\Prb(A_n)$.
		\end{enumerate}
	\end{lem}
	\begin{proof}
		Sei $p$ Zähldichte auf $\Omega$.
		\begin{enumerate}
			\item Nach Definition: $\Prb(\Omega)=\sum_{\omega\in\Omega}p(\omega)=1$
			\item 
		\end{enumerate}
	\end{proof}
%TODO

%VL 24.04.2017
	\setcounter{section}{1}
	\setcounter{subsection}{10}
	
	\begin{exmlist}
		\begin{exm}[Einfache Irrfahrt] Dimension $d$, $N$ Schritte. $\Omega=\{(x_0,x_1,...,x_N):X_j\in\Z^d\forall j, x_0=0,|x_{j+1}-x_j|=1\forall j\}$.\\
			Also ist $|\Omega_N|=(2d)^N$, Setze $p(\omega=\frac{1}{(2d)^N})\forall\omega\in\Omega$.\\\\
			Fragestellungen:
			\begin{enumerate}
				\item $A_N:=\{(x_1,...,x_N)\}\in\Omega_N:\exists j>0$ mit $x_j=0\}$. (``Rückkehr zum Startpunkt'').\\
				Es ist klar, dass $\Prb(A_N)\ge\frac{1}{2d}>0$, falls $N\ge 2$.\\
				Es ist leicht zu zeigen, dass $N\mapsto\mathbb{P}(A_N)$ wächst monoton.\\
				\begin{description}
					\item[Knifflig:] Was ist $\lim\limits_{N\to\infty}$? $<1$? $=1$?
					\item[Antwort:] $=1$ für $d\leq 2$, $<1$ für $d\geq 3$.
				\end{description}
				\item $B_n,\al:=\left\{\omega=(x_0,x_1,...,x_N)\in\Omega_N:|x_N|\ge N^\alpha\right\}$ für $0<\alpha\leq 1$
				\begin{description}
					\item[Frage:] $\lim\limits_{N\to\inf}\Prb(B_{n,\al})$?
					\item[Antwort:] $0$, falls $\alpha>\frac{1}{2}$\\
					$1$, falls $\alpha< \frac{1}{2}$\\
					Für $\alpha=\frac{1}{2}$ gilt 
					\[\lim\limits_{N\to\infty}\Prb(B_{n,\al})=\frac{V_k(d)}{(2\pi)^{\frac{d}{2}}}\int_{1}^{\infty}r^{d-1}\exp(\frac{1}{2}r^2)dr\]
					(dabei ist $V_k(d)$ das Volumen der $d$-Dimensionalen Einheitskugel).
				\end{description}
			\end{enumerate}
		\end{exm}
		\begin{exm}[Selbstvermeidende Irrfahrt] Dimension $d$, $N$ Schritte.
			\begin{enumerate}
				\item $\Omega_N^0=\left\{(x_0,x_1,...,x_N)\in\Omega_N:x_i\neq x_j\text{ falls }i\neq j\right\}$
				Dann gilt für die Anzahl der Pfade:
				\[|\Omega_N^0|=\begin{cases}
				2,&\text{falls $d=1$}\\??,&\text{falls $d>1$}
				\end{cases}\]
				und es ist $p(\omega)=\frac{1}{|\Omega_N^0|\forall\omega\in\Omega_N^0}$.
				\item Wie in a)2. 
				\begin{description}
					\item[Frage] Was ist $\lim\limits_{N\to\infty}\Prb(B_{N,\al}^0)$.
					\item[Bekannt] $\exists\al_c>0$ mit \[\lim\limits_{N\to\infty}\mathbb{ P(B_{n,\al}^0)}=\begin{cases}
					0,&\text{falls $\al>\al_c$}\\1,&\text{falls $\al<\al_c$}
					\end{cases}\]
					\item[Bekannte Werte:]
					\begin{description}
						\item[$d=1$] $\al_c=1$
						\item[$d=2$] $\al_c=\frac{3}{4}$, falls SLE-Conjecture stimmt
						\item[$d=3$] $\al_c\approx 0,5876$ (Numerik)
						\item[$d\ge 4$] $\al_c=\frac{1}{2}$
					\end{description}
				\end{description}
			\end{enumerate}
		\end{exm}
	\end{exmlist}

	\begin{exm}
		Auswählen einer Zufälligen rellen Zahl in $[0,1]$, alle Zahlen sollen di gleich Wahrscheinlichkeit haben:\\
		$[0,1]$ ist nicht endlich, also ist Gleiche Wahrscheinlichkeit für alle Zahlen unmöglich.\\
		$[0,1)$ ist nicht abzählbar, also scheitert der bisherige Ansatz mit der Zähldichte.\\
		\begin{description}
			\item[Ein möglicher Ausweg]
			Definiere $\Prb([a,b])=\Prb((a,b))=\Prb([a,b))=\Prb((a,b])$.\\
			Die Erweiterung, sodass $\forall A\in\potset([0,1])$ $\Prb(A)$ definiert ist, ist nicht möglich.
			\item[Lösung] Definiere $\Prb$ nicht auf allen Mengen $\potset([0,1])$.
		\end{description}
	\end{exm}
	\begin{definition}
		Sei $\Omega$ eine nichtleere Menge.\\
		Ein Mengensystem $\mathscr F\subset \potset(\Omega)$heißt \textbf{$\sigma$-Algebra}, falls
		\begin{enumerate}
			\item $\Omega\in\mathscr F$
			\item Falls $A\in\mathscr F$, dann auch $A^C\in\mathscr F$.
			\item Falls $A_1,A_2,...\in\mathscr F$, dann auch $\bigcap A_i\in\mathscr F$.
		\end{enumerate}
	$(\Omega,\mathscr F)$ heißt dann \textbf{messbarer Raum} oder \textbf{Ereignisraum}.
	\end{definition}

	\begin{bem}
		$\mathscr F$ ist ``die Menge aller Teilmengen von $\Omega$, für die die zugehörige Ja-Nein-Frage beantwortbar ist''.\\
		Daher meint
		\begin{enumerate}
			\item ``Ist $\omega\in\Omega$'' muss beantwortbar sein.
			\item Falls ``Ist $\omega\in A?$'' beantwortbar, so ist auch ``Ist $\omega\notin A?$'' beantwortbar.
			\item Falls ``Ist $\omega\in A_i?$'' beantwortbar für alle $i$, dann ist auch ``Ist $\omega$ in irgendeinem $A_i$?'' beantwortbar.
		\end{enumerate}
	\end{bem}

	\begin{exm}
		Sei $\Omega=[0,1)$, dann ist \begin{enumerate}
			\item $\mathscr F_0:=\{\emptyset,\Omega\}$
			\item $\mathscr F_1:=\{\emptyset,[0,\frac{1}{3}),[\frac{1}{3},1),\Omega\}$.\\
			Die Frage ``Ist $\omega\ge \frac{1}{2}$'' ist hier \underline{nicht} beantwortbar!
			\item $A_{j,n}:=\left[\frac{j}{n},\frac{j+1}{n}\right)$, $n$ ist fest, $j\ge n$.\\
			$\mathscr F_2=\left\{\bigcup_{k=1}^{n}B_{k,n}:B_{k,n}\in\{\emptyset,A_{k,n}\}\right\}$
			\item $\mathscr F_3=\potset(\Omega)$ ist ebenfalls eine $\sigma$-Algebra.
		\end{enumerate}
	\end{exm}
	
	\begin{satz}
		Sei $\mathscr G\subset\potset(\Omega)$ ein Mengensystem. Sei $\Sigma:=\{\mathscr A\subset \potset(\Omega):\text{$\mathscr A$ ist $\sigma$-Algebra und $\mathscr G\subset\mathscr A$}$.\\
		Dann ist auch $\bigcap_{\mathscr A\in\Sigma}\mathscr A$ eine $\sigma$-Algebra.
	\end{satz}
	%TODO counter
	\addtocounter{subsection}{-1}
	\begin{definition}
		$\sigma(\mathscr G):= \bigcap_{\mathscr A\in\Sigma}\mathscr A$ heißt \textbf{die von $\mathscr G$ erzeugt $\sigma$-Algebra}.
	\end{definition}

	\begin{definition}
		Sei $\Omega=\R$, $\mathscr G:=\{[a,b]:a,b\in\R,a<b\}$.\\
		$\mathscr B:=\sigma(\mathscr G)$ heiß \textbf{Borel-$\sigma$-Algebra}.
	\end{definition}

	\begin{bem}
		\begin{enumerate}
			\item $\mathscr B$ enthält alle offenen Mengen, alle abbgeschlossen Mengen und alle halboffenen Intervalle.
			\item $\mathscr B\subsetneqq \potset (\Omega)$.
			\item $\mathscr B$kann nicht abzählbar konstruiert werden.
			\item $\mathscr B=\sigma(\{(-\infty,c]\}:c\in\R)$.
			\item Falls $\Omega_o
			0\subset\R$, $\Omega_0\neq \emptyset$, dann ist
			\[\mathscr B_{\Omega_0}:=\{ A\cap\Omega_0:A\in \mathscr B(\R)\}\]
		eine $\sigma$-Algebra, die \textbf{Einschränkung} von $\mathscr B$ auf $\Omega_0$.
		\end{enumerate}
	\end{bem}

	\begin{definition}
		Seien $E_1,E_2,...,E_N$ Mengen, $N\leq \infty$.\\
		$\mathscr E_i$ seien $\sigma$-Algebren auf $E_i$ und es sei
		\[\Omega=\operatornamewithlimits{X}_{i=1}^NE_i=\left\{(e_1,...,e_N):e_i\in E_i\forall i\leq N\right\}\]
		Eine Menge der Form
		\[A_{j,B_j}=\left\{(e_1,...,e_N):e_j\in B_j,\text{ andere $e_k$ beliebig}\right\}\]
		mit $B_j\in\mathscr E_j,j\leq N$ heißt \textbf{Zylindermenge}.\\
	\end{definition}
	%TODO COUNTER
	\addtocounter{subsection}{-1}
	\begin{definition}
		Die $\sigma$-Algebra in $\Omega$ die von allen Zylindermengen Erzeugt wird heißt \textbf{Produkt-$\sigma$-Algebra}.
		Man nennt $\mathscr Z$ das System der Zylindermenge und $\mathscr E_1\otimes\mathscr E_2\otimes...\otimes \mathscr E_N:=\sigma(\mathscr Z)$.
	\end{definition}

%VL 02.05.2017
\setcounter{subsection}{26}

	\begin{definition}
		Die Abbildung $\la:\Bor(\R^n)\to[0,\infty], A\mapsto\int_{\R^n}\chi_A(x)~dx=\la(A)$ heißt \textbf{Lebesgue-Maß}
	\end{definition}
	
	\begin{exm}\label{0128exm}
			Sei $\varrho:\R^n\to[0,\infty)$ Borel-messbar und $\int\varrho(x)~dx=1$.\\
			Dann ist die Abbildung $\Prb_p\varrho:\Bor(\R^n)\to[0,1],A\mapsto\int_A\varrho(x)~dx=\int\chi_A(y)\varrho(x)~dx$ ein Wahrscheinlichkeitsmaß.
	\end{exm}

	\begin{definition}
		Sei $\varrho:\R^n\to[0,\infty)$ Borel-messbar und $\int\varrho(x)~dx=1$, (=\ref{0128exm}) dann heißt $\varrho$ \textbf{Dichte} von $\Prb_p\varrho$.
	\end{definition}

	\begin{exm}
		Sei $\varrho$ eine Dichte, $x\in\R$. Dann hat $\Prb:=\frac{1}{3}\Prb_p\varrho+\frac{2}{3}\delta_x$ keine Dichte (siehe \ref{0121exm})
	\end{exm}

	\begin{bem*}
		Wenn $\varrho$ Dichte ist schreibt man auch $\Prb_p\varrho\equiv\varrho(x)~dx$
	\end{bem*}

	\begin{definition}
		Sei $\Omega\in\Bor(\R^n)$ mit $\la(\Omega)<\infty$. Das Wahrscheinlichkeitsmaß auf $\Omega$ mit Dichte $\varrho(y)=\frac{1}{\la(\Omega)}$ heißt \textbf{Gleichverteilung} auf $\Omega$.\\
		Man fasst dann $\tilde\varrho(x)=\frac{1}{\la(\Omega)}\chi_A\Omega(x)$ als Einbettung in den $\R^n$ auf.
	\end{definition}

	\begin{rem*}[Zufallsvariable]
		Der Begriff \enquote{Zufallsvariable} ist historisch gewachsen. (Keine Variable einer Funktion).
		\begin{description}
			\item[Problemstellung] Von einem komplizierten Zufälligen Geschehen will man nur gewisse Aspekte betrachten.
		\end{description}
	\end{rem*}

	\begin{exmlist}
		\begin{exm}[2 mal Würfeln, Würfelsumme]
			Sei $\Omega:\{1,2,3,4,5,6\}\times\{1,2,4,5,6\}$. $\omega=(\omega_1,\omega_2)$. Der zu betrachtende Aspekt: $S(\omega_1,\omega_2)=\omega_1+\omega_2\in\{2,3,...,12\}\neq \Omega$.
		\end{exm}
		\begin{exm}
			Sei $\Omega=\Omega_N$(siehe \ref{0111aexm}, einfache Irrfahrt).\\
			$\omega=(\omega_1,\omega_2,...\omega_N)\in(\Z^d)^N$.\begin{description}
				\item[Aspekt 1] Position nach $N$ Schritten.\\
				Modell: $X_N\big((\omega_1,...,\omega_N)\big)=\omega_N\in\Z^d\neq \Omega$
				\item[Aspekt 2] Maximaler Abstand vom Ursprung bis zum Schritt $N$.\\
				Modell $M_N\big((\omega_1,...,\omega_N)\big)=\max\{|\omega_j|,j\le N\}$.
			\end{description} 
		\end{exm}
		\begin{exm}
			Sei $\Omega=[0,1]$, $\F\in\Bor([0,1])$, $\omega=x\in[0,1]$.
			\begin{description}
				\item[Aspekt 1] Erste Ziffer nach dem Komma?\\
				Modell: $y_1(x)=\left\lfloor10x\right\rfloor$
				\item[Aspekt 2] Fläche des Quadrates mit Kantenlänge $x$\\
				Modell: $Q(x)=x^2\in[0,1]$.
			\end{description}
		\end{exm}
	Fazit: Modellierung durch Abbildungen.
	\end{exmlist}

	\begin{definition}
		Seine $(\Omega,\F)$ und $(\Omega',\F')$ Ereignisräume.\\
		eine Abbildung $X:\Omega\to\Omega'$ heißt \textbf{Zufallsvariable} (ZV) [oder messbare Abbildung, zufälliges Element von $\Omega'$], falls gilt:\\
		$\forall A'\in\F'$ ist $X^{-1}(A')\in\F$.\\
		Hierbei ist $X^{-1}$ das Urbild von $A'$ unter $X$.
	\end{definition}

	\begin{bem}
		Die Urbild-Abbildung bilte Mengen in $\F'$ (d.h.erlaube Ja-Nein-Fragen) auf Mengen  in $\potset(\Omega)$ (d.h. Ja-Nein-Fragen) ab.
	\end{bem}

	\begin{exm*}
		In 1.32.1 \ref{0132a} ist $S^{-1}(\{4\})=\{(1,3),(2,2),(3,1)\}$.\\
		In 1.32.1 \ref{0132a} ist $Y_1^{-1}(\{3,7\})=[0,3,0,4)\cup[0,7,0,8)$ %TODO ???
	\end{exm*}

	\begin{bem*}
		Die Bedingung (*) bedeutet, dass für alle durch $A\in\F'$ erzeugte erlaubten Ja-Nein-Fragen auch die Frage \enquote{Liegt $X(\omega)$ in $A'$?} erlaubt ist.
	\end{bem*}
	\begin{bem*}
		Oft nimmar  man nur $\Omega$ und $(\Omega',\F')$ als gegeben.\\
		Dann ist $X^{-1}(\F'):=\{X^{-1}(A')\mid A'\in \F'\}$ die von $X$ erzeugt $\sigma$-Algebra.
	\end{bem*}
	\begin{bem*}
		Falls $\F=\potset(\Omega)$, dann ist jede Abbildung eine Zufallsvariable.
	\end{bem*}

	\begin{lem}\label{0135lem}
		Seine $(\Omega,\F)$ und $(\Omega',\F')$ Ereignisräume, $X:\Omega\to\Omega'$ und sei $\mathscr G'$ ein Mengensystem mit $\F'=\sigma(G')$. Dann ist $X$ genau dann Zufallsvariable, wenn $X^{-1}(A')\in\F\forall A'\in \mathscr G'$.
	\end{lem}
	\begin{proof}
		\begin{description}
			\item[\enquote{$\Rightarrow$}] ist klar, da $\mathscr G'\subset \F'$
			\item[\enquote{$\Leftarrow$}] Sei $\mathscr A':=\{A'\in\Omega'\mid X^{-1}(A')\in\F\}$ ist eine $\sigma$-Algebra und $\mathscr A'\supset\mathscr G'$ nach Annahme.\\
			Daher ist $\F'=\sigma(\mathscr G')\subset\mathscr A'$, sodass $X^{-1}(A')\in\F\forall A\in\F$.
		\end{description}
	\end{proof}

	\begin{exmlist}
		\begin{exm}
			Sei $(\Omega',\F')=(\R,\R)$. Nach \ref{0135lem} gilt:\\
			$X:\Omega\to\Omega'$ ist genau dann Zufallsvariable, wenn $X^{-1}\big((-\infty,c)\big)\in\F\forall c\in\R$
			Für $\Omega'=\R$ heißt $X$ \textbf{relle Zufallsvariable}
		\end{exm}
		\begin{exm}
			Es ist $\ol{\R}:=[-\infty,\infty]$ mit $\sigma$-Algebra $\Bor(\ol{\R})=\sigma\big(\{[-\infty-c]:c\in\ol{\R}\}\big)$.\\
			Die Abbildung $X:\Omega\to\ol{\R}$ ist genau dann Zufallsvariable, wenn $X^{-1}\big([-\infty,c]\big)\in\F\forall c$.\\
			Dann heißt $X$ \textbf{numerische Zufallsvariable}
		\end{exm}
	\end{exmlist}

	\begin{theorem}
		Sei $(\Omega,\F,\Prb)$ ein Wahrscheinlichkeitsraum, $(\Omega',\F')$ ein Ereignisraum, $X:\Omega\to\Omega'$ eine Zufallsvariable.\\
		Dann ist die Abbildung
		\[\Prb':\F'\to[0,1],\quad A'\mapsto\Prb'(A'):=\Prb(X^{-1}('A))\] 
		ein Wahrscheinlichkeitsmaß auf $(\Omega',\F')$.  
	\end{theorem}
	%TODO COUNTER
	\addtocounter{subsection}{-1}
	\begin{definition}
		$\Prb'$ heißt \textbf{Bildmaß von $\Prb$ unter $X$} oder \textbf{Verteilung von $X$ unter $\Prb$}.\\
		Man schreibt $\Prb'=\Prb\circ X^{-1}$ oder $\Prb'=\Prb_X$
	\end{definition}
	\begin{proof}
		Da $X$ eine Zufallsvariable ist, ist $X^{-1}(A')\in\F\forall A'\in\F'$, daher im Definitionsbereich von $\Prb$.\\
		Also ist $\Prb'$ wohldefiniert. Prüfe Definition 1.20 \ref{0112def}.
		\begin{enumerate}[label=\alph*)]
			\item $\Prb'(\Omega')=\Prb(X^{-1}(\Omega'))=\Prb(\Omega)=1$
			\item $A_1',A_2',...\in \F'$ seien paarweise disjunkt. Dann sind $X^{-1}(A_1'),X^{-1}(A_2'),...$ auch paarweise disjunkt.
			\begin{align*}
			\Prb'\left(\bigcup_{i=1}^\infty A_i'\right)&=\Prb\left( X^{-1}\left(\bigcup_{i=1}^\infty A_i'\right)\right)\\
			&=\Prb\left(\bigcup_{i=1}^\infty X^{-1}(A_i')\right)\\
			&=\sum_{i=1}\infty\Prb(X^{-1}(A_i'))\\
			&=\sum_{i=1}\infty\Prb'(A_i')
			\end{align*}
		\end{enumerate}
	\end{proof}

	\begin{definition}
		Seien $(\Omega_1,\F_1,\Prb_1)$ und $(\Omega_2,\F_2,\Prb_2)$ Wahrscheinlichkeitsräume, $X_1:\Omega_1\to\Omega'_1$, $X_2:\Omega_2\to\Omega'_2$ Zufallsvariablen.\\
		Falls $\Prb_1(X^{-1}_1(A'))=\Prb_2(X_2^{-1}(A'))\forall A'\in\F'$, dann heißen $X_1$ und $X_2$ identisch verteilt.
	\end{definition}

	\begin{bem}[Notation]
		Man schreibt oft:
		\begin{itemize}
			\item $\{X\in A'\}$ statt $X^{-1}(A')$
			\item $\Prb(\{X\in A'\})$ oder $\Prb(X\in A')$ statt $\Prb(X^{-1}(A'))$
			\item $\Prb_X$ statt $\Prb\circ X^{-1}$.
		\end{itemize}
	\end{bem}
\end{document}
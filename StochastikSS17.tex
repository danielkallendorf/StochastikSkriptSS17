\documentclass[10pt,a4paper]{article}

\usepackage{luatex85}
\def\pgfsysdriver{pgfsys-pdftex.def}

%\usepackage[utf8]{inputenc}
%\usepackage{fontenc}
\usepackage[utf8]{luainputenc}

\usepackage[german]{babel}
\usepackage{amsmath}
\usepackage{amsfonts}
\usepackage{amssymb}
\usepackage{amsthm}
\usepackage{graphicx}
\usepackage{tikz,pgf}
\usetikzlibrary{cd}
\usetikzlibrary{babel}
\usepackage{mathrsfs}
\usepackage{framed}
\usepackage[hidelinks]{hyperref}
\usepackage{ulem}


\newcommand{\N}{\ensuremath{\mathbb{N}}}
\newcommand{\Z}{\ensuremath{\mathbb{Z}}}
\newcommand{\Q}{\ensuremath{\mathbb{Q}}}
\newcommand{\R}{\ensuremath{\mathbb{R}}}
\newcommand{\C}{\ensuremath{\mathbb{C}}}
\newcommand{\F}{\ensuremath{\mathbb{F}}}
\newcommand{\AF}{\ensuremath{\mathbb{A}}}

\newcommand{\la}{\ensuremath{\lambda}}
\newcommand{\al}{\ensuremath{\alpha}}
\newcommand{\ol}[1]{\overline{#1}}
\newcommand{\ul}[1]{\underline{#1}}
\newcommand{\todomark}[1]{\fbox{\Large Hier könnte \sout{Ihre Werbung} #1 stehen}}
\newcommand{\norm}[1]{\left|#1\right|}
\newcommand{\mapsfrom}{\ensuremath{\mathrel{\reflectbox{\mapsto}}}}
\newcommand{\potset}{\mathscr P}
\newcommand{\Prb}{\mathbb P}


\newcounter{thm}[section]
\renewcommand{\thethm}{\arabic{section}.\arabic{thm}}
\renewcommand{\thesubsection}{\arabic{section}.\arabic{thm}}

\let\oldsubsection\subsection
\renewcommand{\subsection}{\stepcounter{thm}\oldsubsection}

\theoremstyle{definition}
\newtheorem{definition}[thm]{Definition}
\newtheorem{prop}[thm]{Proposition}

\theoremstyle{plain}
\newtheorem{lem}[thm]{Lemma}
\newtheorem{kor}[thm]{Korollar}
\newtheorem{satz}[thm]{Satz}
\newtheorem{theorem}[thm]{Theorem}

\theoremstyle{remark}
\newtheorem{bem}[thm]{Bemerkung}
\newtheorem*{bem*}{Bemerkung}
\newtheorem{rem}[thm]{Erinnerung}
\newtheorem{exm}[thm]{Beispiel}
\newtheorem*{exm*}{Beispiel}

\title{Einführung in die Stochastik}
\author{}

\begin{document}
	\maketitle
	\tableofcontents
	
%VL 24.04.2017
	\setcounter{section}{1}
	\setcounter{thm}{10}
	\subsection{Beispiel}
	\paragraph*{a) Einfache Irrfahrt} Dimension $d$, $N$ Schritte. $\Omega=\{(x_0,x_1,...,x_N):X_j\in\Z^d\forall j, x_0=0,|x_{j+1}-x_j|=1\forall j\}$.\\
	Also ist $|\Omega_N|=(2d)^N$, Setze $p(\omega=\frac{1}{(2d)^N})\forall\omega\in\Omega$.
	\subsubsection*{Fragestellungen:}
	\begin{enumerate}
		\item $A_N:=\{(x_1,...,x_N)\}\in\Omega_N:\exists j>0$ mit $x_j=0\}$. (``Rückkehr zum Startpunkt'').\\
		Es ist klar, dass $\Prb(A_N)\ge\frac{1}{2d}>0$, falls $N\ge 2$.\\
		Es ist leicht zu zeigen, dass $N\mapsto\mathbb{P}(A_N)$ wächst monoton.\\
		\begin{description}
			\item[Knifflig:] Was ist $\lim\limits_{N\to\infty}$? $<1$? $=1$?
			\item[Antwort:] $=1$ für $d\leq 2$, $<1$ für $d\geq 3$.
		\end{description}
		\item $B_n,\al:=\left\{\omega=(x_0,x_1,...,x_N)\in\Omega_N:|x_N|\ge N^\alpha\right\}$ für $0<\alpha\leq 1$
		\begin{description}
			\item[Frage:] $\lim\limits_{N\to\inf}\Prb(B_{n,\al})$?
			\item[Antwort:] $0$, falls $\alpha>\frac{1}{2}$\\
			$1$, falls $\alpha< \frac{1}{2}$\\
			Für $\alpha=\frac{1}{2}$ gilt 
			\[\lim\limits_{N\to\infty}\Prb(B_{n,\al})=\frac{V_k(d)}{(2\pi)^{\frac{d}{2}}}\int_{1}^{\infty}r^{d-1}\exp(\frac{1}{2}r^2)dr\]
			(dabei ist $V_k(d)$ das Volumen der $d$-Dimensionalen Einheitskugel).
		\end{description}
	\end{enumerate}
	\paragraph*{b) Selbstvermeidende Irrfahrt}Dimension $d$, $N$ Schritte.
	\begin{enumerate}
		\item $\Omega_N^0=\left\{(x_0,x_1,...,x_N)\in\Omega_N:x_i\neq x_j\text{ falls }i\neq j\right\}$
		Dann gilt für die Anzahl der Pfade:
		\[|\Omega_N^0|=\begin{cases}
		2,&\text{falls $d=1$}\\??,&\text{falls $d>1$}
		\end{cases}\]
		und es ist $p(\omega)=\frac{1}{|\Omega_N^0|\forall\omega\in\Omega_N^0}$.
		\item Wie in a)2. 
		\begin{description}
			\item[Frage] Was ist $\lim\limits_{N\to\infty}\Prb(B_{N,\al}^0)$.
			\item[Bekannt] $\exists\al_c>0$ mit \[\lim\limits_{N\to\infty}\mathbb{ P(B_{n,\al}^0)}=\begin{cases}
			0,&\text{falls $\al>\al_c$}\\1,&\text{falls $\al<\al_c$}
			\end{cases}\]
			\item[Bekannte Werte:]
			\begin{description}
				\item[$d=1$] $\al_c=1$
				\item[$d=2$] $\al_c=\frac{3}{4}$, falls SLE-Conjecture stimmt
				\item[$d=3$] $\al_c\approx 0,5876$ (Numerik)
				\item[$d\ge 4$] $\al_c=\frac{1}{2}$
			\end{description}
		\end{description}
	\end{enumerate}
	\subsection{Beispiel}
	Auswählen einer Zufälligen rellen Zahl in $[0,1]$, alle Zahlen sollen di gleich Wahrscheinlichkeit haben:\\
	$[0,1]$ ist nicht endlich, also ist Gleiche Wahrscheinlichkeit für alle Zahlen unmöglich.\\
	$[0,1)$ ist nicht abzählbar, also scheitert der bisherige Ansatz mit der Zähldichte.\\
	\paragraph*{Ein möglicher Ausweg} Definiere $\Prb([a,b])=\Prb((a,b))=\Prb([a,b))=\Prb((a,b])$.\\
	Die Erweiterung, sodass $\forall A\in\potset([0,1])$ $\Prb(A)$ definiert ist, ist nicht möglich.
	\paragraph*{Lösung:} Definiere $\Prb$ nicht auf allen Mengen $\potset([0,1])$.
	
	\begin{definition}
		Sei $\Omega$ eine nichtleere Menge.\\
		Ein Mengensystem $\mathscr F\subset \potset(\Omega)$heißt \textbf{$\sigma$-Algebra}, falls
		\begin{enumerate}
			\item $\Omega\in\mathscr F$
			\item Falls $A\in\mathscr F$, dann auch $A^C\in\mathscr F$.
			\item Falls $A_1,A_2,...\in\mathscr F$, dann auch $\bigcap A_i\in\mathscr F$.
		\end{enumerate}
	$(\Omega,\mathscr F)$ heißt dann \textbf{messbarer Raum} oder \textbf{Ereignisraum}.
	\end{definition}

	\begin{bem}
		$\mathscr F$ ist ``die Menge aller Teilmengen von $\Omega$, für die die zugehörige Ja-Nein-Frage beantwortbar ist''.\\
		Daher meint
		\begin{enumerate}
			\item ``Ist $\omega\in\Omega$'' muss beantwortbar sein.
			\item Falls ``Ist $\omega\in A?$'' beantwortbar, so ist auch ``Ist $\omega\notin A?$'' beantwortbar.
			\item Falls ``Ist $\omega\in A_i?$'' beantwortbar für alle $i$, dann ist auch ``Ist $\omega$ in irgendeinem $A_i$?'' beantwortbar.
		\end{enumerate}
	\end{bem}

	\begin{exm}
		Sei $\Omega=[0,1)$, dann ist \begin{enumerate}
			\item $\mathscr F_0:=\{\emptyset,\Omega\}$
			\item $\mathscr F_1:=\{\emptyset,[0,\frac{1}{3}),[\frac{1}{3},1),\Omega\}$.\\
			Die Frage ``Ist $\omega\ge \frac{1}{2}$'' ist hier \underline{nicht} beantwortbar!
			\item $A_{j,n}:=\left[\frac{j}{n},\frac{j+1}{n}\right)$, $n$ ist fest, $j\ge n$.\\
			$\mathscr F_2=\left\{\bigcup_{k=1}^{n}B_{k,n}:B_{k,n}\in\{\emptyset,A_{k,n}\}\right\}$
			\item $\mathscr F_3=\potset(\Omega)$ ist ebenfalls eine $\sigma$-Algebra.
		\end{enumerate}
	\end{exm}
	
	\begin{satz}
		Sei $\mathscr G\subset\potset(\Omega)$ ein Mengensystem. Sei $\Sigma:=\{\mathscr A\subset \potset(\Omega):\text{$\mathscr A$ ist $\sigma$-Algebra und $\mathscr G\subset\mathscr A$}$.\\
		Dann ist auch $\bigcap_{\mathscr A\in\Sigma}\mathscr A$ eine $\sigma$-Algebra.
	\end{satz}
	%TODO counter
	\addtocounter{thm}{-1}
	\begin{definition}
		$\sigma(\mathscr G):= \bigcap_{\mathscr A\in\Sigma}\mathscr A$ heißt \textbf{die von $\mathscr G$ erzeugt $\sigma$-Algebra}.
	\end{definition}

	\begin{definition}
		Sei $\Omega=\R$, $\mathscr G:=\{[a,b]:a,b\in\R,a<b\}$.\\
		$\mathscr B:=\sigma(\mathscr G)$ heiß \textbf{Borel-$\sigma$-Algebra}.
	\end{definition}

	\begin{bem}
		\begin{enumerate}
			\item $\mathscr B$ enthält alle offenen Mengen, alle abbgeschlossen Mengen und alle halboffenen Intervalle.
			\item $\mathscr B\subsetneqq \potset (\Omega)$.
			\item $\mathscr B$kann nicht abzählbar konstruiert werden.
			\item $\mathscr B=\sigma(\{(-\infty,c]\}:c\in\R)$.
			\item Falls $\Omega_o
			0\subset\R$, $\Omega_0\neq \emptyset$, dann ist
			\[\mathscr B_{\Omega_0}:=\{ A\cap\Omega_0:A\in \mathscr B(\R)\}\]
		eine $\sigma$-Algebra, die \textbf{Einschränkung} von $\mathscr B$ auf $\Omega_0$.
		\end{enumerate}
	\end{bem}

	\begin{definition}
		Seien $E_1,E_2,...,E_N$ Mengen, $N\leq \infty$.\\
		$\mathscr E_i$ seien $\sigma$-Algebren auf $E_i$ und es sei
		\[\Omega=\operatornamewithlimits{X}_{i=1}^NE_i=\left\{(e_1,...,e_N):e_i\in E_i\forall i\leq N\right\}\]
		Eine Menge der Form
		\[A_{j,B_j}=\left\{(e_1,...,e_N):e_j\in B_j,\text{ andere $e_k$ beliebig}\right\}\]
		mit $B_j\in\mathscr E_j,j\leq N$ heißt \textbf{Zylindermenge}.\\
	\end{definition}
	%TODO COUNTER
	\addtocounter{thm}{-1}
	\begin{definition}
		Die $\sigma$-Algebra in $\Omega$ die von allen Zylindermengen Erzeugt wird heißt \textbf{Produkt-$\sigma$-Algebra}.
		Man nennt $\mathscr Z$ das System der Zylindermenge und $\mathscr E_1\otimes\mathscr E_2\otimes...\otimes \mathscr E_N:=\sigma(\mathscr Z)$.
	\end{definition}
\end{document}